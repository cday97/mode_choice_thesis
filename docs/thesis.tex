% This is the Reed College LaTeX thesis template. Most of the work
% for the document class was done by Sam Noble (SN), as well as this
% template. Later comments etc. by Ben Salzberg (BTS). Additional
% restructuring and APA support by Jess Youngberg (JY).
% Your comments and suggestions are more than welcome; please email
% them to cus@reed.edu
%
% See https://www.reed.edu/cis/help/LaTeX/index.html for help. There are a
% great bunch of help pages there, with notes on
% getting started, bibtex, etc. Go there and read it if you're not
% already familiar with LaTeX.
%
% Any line that starts with a percent symbol is a comment.
% They won't show up in the document, and are useful for notes
% to yourself and explaining commands.
% Commenting also removes a line from the document;
% very handy for troubleshooting problems. -BTS

% As far as I know, this follows the requirements laid out in
% the 2002-2003 Senior Handbook. Ask a librarian to check the
% document before binding. -SN

%%
%% Preamble
%%
% \documentclass{<something>} must begin each LaTeX document
\documentclass[12pt, oneside, openright]{byuthesis}


\title{Building a Thesis in \texttt{Bookdown} and LaTeX, which can have a\\
Wrapped Title if Long}
\author{Cosmo Cougar}
\copyyear{20XX}
\committeeMembers{Brigham Young, Chair

Karl G. Maeser

Harvey Fletcher}
\keywords{engineering, optimization, document creation}
\degree{Master of Science}
\doctype{thesis}
\department{Department of Civil and Construction Engineering}



% End of CII addition
%%
%% End Preamble
%%
%

% Pandoc citation processing
\newlength{\cslhangindent}
\setlength{\cslhangindent}{1.5em}
\newlength{\csllabelwidth}
\setlength{\csllabelwidth}{3em}
\newlength{\cslentryspacingunit} % times entry-spacing
\setlength{\cslentryspacingunit}{\parskip}
% for Pandoc 2.8 to 2.10.1
\newenvironment{cslreferences}%
  {}%
  {\par}
% For Pandoc 2.11+
\newenvironment{CSLReferences}[2] % #1 hanging-ident, #2 entry spacing
 {% don't indent paragraphs
  \setlength{\parindent}{0pt}
  % turn on hanging indent if param 1 is 1
  \ifodd #1
  \let\oldpar\par
  \def\par{\hangindent=\cslhangindent\oldpar}
  \fi
  % set entry spacing
  \setlength{\parskip}{#2\cslentryspacingunit}
 }%
 {}
\usepackage{calc}
\newcommand{\CSLBlock}[1]{#1\hfill\break}
\newcommand{\CSLLeftMargin}[1]{\parbox[t]{\csllabelwidth}{#1}}
\newcommand{\CSLRightInline}[1]{\parbox[t]{\linewidth - \csllabelwidth}{#1}\break}
\newcommand{\CSLIndent}[1]{\hspace{\cslhangindent}#1}

\providecommand{\tightlist}{%
  \setlength{\itemsep}{0pt}\setlength{\parskip}{0pt}}

\usepackage{color}
\usepackage{fancyvrb}
\newcommand{\VerbBar}{|}
\newcommand{\VERB}{\Verb[commandchars=\\\{\}]}
\DefineVerbatimEnvironment{Highlighting}{Verbatim}{commandchars=\\\{\}}
% Add ',fontsize=\small' for more characters per line
\usepackage{framed}
\definecolor{shadecolor}{RGB}{248,248,248}
\newenvironment{Shaded}{\begin{snugshade}}{\end{snugshade}}
\newcommand{\AlertTok}[1]{\textcolor[rgb]{0.94,0.16,0.16}{#1}}
\newcommand{\AnnotationTok}[1]{\textcolor[rgb]{0.56,0.35,0.01}{\textbf{\textit{#1}}}}
\newcommand{\AttributeTok}[1]{\textcolor[rgb]{0.77,0.63,0.00}{#1}}
\newcommand{\BaseNTok}[1]{\textcolor[rgb]{0.00,0.00,0.81}{#1}}
\newcommand{\BuiltInTok}[1]{#1}
\newcommand{\CharTok}[1]{\textcolor[rgb]{0.31,0.60,0.02}{#1}}
\newcommand{\CommentTok}[1]{\textcolor[rgb]{0.56,0.35,0.01}{\textit{#1}}}
\newcommand{\CommentVarTok}[1]{\textcolor[rgb]{0.56,0.35,0.01}{\textbf{\textit{#1}}}}
\newcommand{\ConstantTok}[1]{\textcolor[rgb]{0.00,0.00,0.00}{#1}}
\newcommand{\ControlFlowTok}[1]{\textcolor[rgb]{0.13,0.29,0.53}{\textbf{#1}}}
\newcommand{\DataTypeTok}[1]{\textcolor[rgb]{0.13,0.29,0.53}{#1}}
\newcommand{\DecValTok}[1]{\textcolor[rgb]{0.00,0.00,0.81}{#1}}
\newcommand{\DocumentationTok}[1]{\textcolor[rgb]{0.56,0.35,0.01}{\textbf{\textit{#1}}}}
\newcommand{\ErrorTok}[1]{\textcolor[rgb]{0.64,0.00,0.00}{\textbf{#1}}}
\newcommand{\ExtensionTok}[1]{#1}
\newcommand{\FloatTok}[1]{\textcolor[rgb]{0.00,0.00,0.81}{#1}}
\newcommand{\FunctionTok}[1]{\textcolor[rgb]{0.00,0.00,0.00}{#1}}
\newcommand{\ImportTok}[1]{#1}
\newcommand{\InformationTok}[1]{\textcolor[rgb]{0.56,0.35,0.01}{\textbf{\textit{#1}}}}
\newcommand{\KeywordTok}[1]{\textcolor[rgb]{0.13,0.29,0.53}{\textbf{#1}}}
\newcommand{\NormalTok}[1]{#1}
\newcommand{\OperatorTok}[1]{\textcolor[rgb]{0.81,0.36,0.00}{\textbf{#1}}}
\newcommand{\OtherTok}[1]{\textcolor[rgb]{0.56,0.35,0.01}{#1}}
\newcommand{\PreprocessorTok}[1]{\textcolor[rgb]{0.56,0.35,0.01}{\textit{#1}}}
\newcommand{\RegionMarkerTok}[1]{#1}
\newcommand{\SpecialCharTok}[1]{\textcolor[rgb]{0.00,0.00,0.00}{#1}}
\newcommand{\SpecialStringTok}[1]{\textcolor[rgb]{0.31,0.60,0.02}{#1}}
\newcommand{\StringTok}[1]{\textcolor[rgb]{0.31,0.60,0.02}{#1}}
\newcommand{\VariableTok}[1]{\textcolor[rgb]{0.00,0.00,0.00}{#1}}
\newcommand{\VerbatimStringTok}[1]{\textcolor[rgb]{0.31,0.60,0.02}{#1}}
\newcommand{\WarningTok}[1]{\textcolor[rgb]{0.56,0.35,0.01}{\textbf{\textit{#1}}}}

\begin{document}

\begin{abstract}
This document describes the use of a new \LaTeX ~template for writing masters
theses at Brigham Young University. This was made necessary by the relative age
and complication of prior templates. Additionally, the new \texttt{cosmodown} R package
contains an implementation of this thesis in R markdown, with the help of the
\href{https://github.com/ismayc/thesisdown}{\texttt{thesisdown}} package.

The abstract of a thesis should describe the
motivation, objective, overall results, and central findings of the thesis.
It may have multiple paragraphs if necessary.
\end{abstract}


\makefrontmatter % this stuff will be roman-numbered
\thesisbody

\hypertarget{about}{%
\chapter{About}\label{about}}

This is a \emph{sample} book written in \textbf{Markdown}. You can use anything that Pandoc's Markdown supports; for example, a math equation \(a^2 + b^2 = c^2\).

\hypertarget{usage}{%
\section{Usage}\label{usage}}

Each \textbf{bookdown} chapter is an .Rmd file, and each .Rmd file can contain one (and only one) chapter. A chapter \emph{must} start with a first-level heading: \texttt{\#\ A\ good\ chapter}, and can contain one (and only one) first-level heading.

Use second-level and higher headings within chapters like: \texttt{\#\#\ A\ short\ section} or \texttt{\#\#\#\ An\ even\ shorter\ section}.

The \texttt{index.Rmd} file is required, and is also your first book chapter. It will be the homepage when you render the book.

\hypertarget{render-book}{%
\section{Render book}\label{render-book}}

You can render the HTML version of this example book without changing anything:

\begin{enumerate}
\def\labelenumi{\arabic{enumi}.}
\item
  Find the \textbf{Build} pane in the RStudio IDE, and
\item
  Click on \textbf{Build Book}, then select your output format, or select ``All formats'' if you'd like to use multiple formats from the same book source files.
\end{enumerate}

Or build the book from the R console:

\begin{Shaded}
\begin{Highlighting}[]
\NormalTok{bookdown}\SpecialCharTok{::}\FunctionTok{render\_book}\NormalTok{()}
\end{Highlighting}
\end{Shaded}

To render this example to PDF as a \texttt{bookdown::pdf\_book}, you'll need to install XeLaTeX. You are recommended to install TinyTeX (which includes XeLaTeX): \url{https://yihui.org/tinytex/}.

\hypertarget{preview-book}{%
\section{Preview book}\label{preview-book}}

As you work, you may start a local server to live preview this HTML book. This preview will update as you edit the book when you save individual .Rmd files. You can start the server in a work session by using the RStudio add-in ``Preview book'', or from the R console:

\begin{Shaded}
\begin{Highlighting}[]
\NormalTok{bookdown}\SpecialCharTok{::}\FunctionTok{serve\_book}\NormalTok{()}
\end{Highlighting}
\end{Shaded}

\hypertarget{hello-bookdown}{%
\chapter{Hello bookdown}\label{hello-bookdown}}

All chapters start with a first-level heading followed by your chapter title, like the line above. There should be only one first-level heading (\texttt{\#}) per .Rmd file.

\hypertarget{a-section}{%
\section{A section}\label{a-section}}

All chapter sections start with a second-level (\texttt{\#\#}) or higher heading followed by your section title, like the sections above and below here. You can have as many as you want within a chapter.

\hypertarget{an-unnumbered-section}{%
\subsection*{An unnumbered section}\label{an-unnumbered-section}}
\addcontentsline{toc}{subsection}{An unnumbered section}

Chapters and sections are numbered by default. To un-number a heading, add a \texttt{\{.unnumbered\}} or the shorter \texttt{\{-\}} at the end of the heading, like in this section.

\hypertarget{cross}{%
\chapter{Cross-references}\label{cross}}

Cross-references make it easier for your readers to find and link to elements in your book.

\hypertarget{chapters-and-sub-chapters}{%
\section{Chapters and sub-chapters}\label{chapters-and-sub-chapters}}

There are two steps to cross-reference any heading:

\begin{enumerate}
\def\labelenumi{\arabic{enumi}.}
\tightlist
\item
  Label the heading: \texttt{\#\ Hello\ world\ \{\#nice-label\}}.

  \begin{itemize}
  \tightlist
  \item
    Leave the label off if you like the automated heading generated based on your heading title: for example, \texttt{\#\ Hello\ world} = \texttt{\#\ Hello\ world\ \{\#hello-world\}}.
  \item
    To label an un-numbered heading, use: \texttt{\#\ Hello\ world\ \{-\#nice-label\}} or \texttt{\{\#\ Hello\ world\ .unnumbered\}}.
  \end{itemize}
\item
  Next, reference the labeled heading anywhere in the text using \texttt{\textbackslash{}@ref(nice-label)}; for example, please see Chapter \ref{cross}.

  \begin{itemize}
  \tightlist
  \item
    If you prefer text as the link instead of a numbered reference use: \protect\hyperlink{cross}{any text you want can go here}.
  \end{itemize}
\end{enumerate}

\hypertarget{captioned-figures-and-tables}{%
\section{Captioned figures and tables}\label{captioned-figures-and-tables}}

Figures and tables \emph{with captions} can also be cross-referenced from elsewhere in your book using \texttt{\textbackslash{}@ref(fig:chunk-label)} and \texttt{\textbackslash{}@ref(tab:chunk-label)}, respectively.

See Figure \ref{fig:nice-fig}.

\begin{Shaded}
\begin{Highlighting}[]
\FunctionTok{par}\NormalTok{(}\AttributeTok{mar =} \FunctionTok{c}\NormalTok{(}\DecValTok{4}\NormalTok{, }\DecValTok{4}\NormalTok{, .}\DecValTok{1}\NormalTok{, .}\DecValTok{1}\NormalTok{))}
\FunctionTok{plot}\NormalTok{(pressure, }\AttributeTok{type =} \StringTok{\textquotesingle{}b\textquotesingle{}}\NormalTok{, }\AttributeTok{pch =} \DecValTok{19}\NormalTok{)}
\end{Highlighting}
\end{Shaded}

\begin{figure}

{\centering \includegraphics[width=0.8\linewidth]{thesis_files/figure-latex/nice-fig-1} 

}

\caption{Here is a nice figure!}\label{fig:nice-fig}
\end{figure}

Don't miss Table \ref{tab:nice-tab}.

\begin{Shaded}
\begin{Highlighting}[]
\NormalTok{knitr}\SpecialCharTok{::}\FunctionTok{kable}\NormalTok{(}
  \FunctionTok{head}\NormalTok{(pressure, }\DecValTok{10}\NormalTok{), }\AttributeTok{caption =} \StringTok{\textquotesingle{}Here is a nice table!\textquotesingle{}}\NormalTok{,}
  \AttributeTok{booktabs =} \ConstantTok{TRUE}
\NormalTok{)}
\end{Highlighting}
\end{Shaded}

\begin{table}

\caption{\label{tab:nice-tab}Here is a nice table!}
\centering
\begin{tabular}[t]{rr}
\toprule
temperature & pressure\\
\midrule
0 & 0.0002\\
20 & 0.0012\\
40 & 0.0060\\
60 & 0.0300\\
80 & 0.0900\\
\addlinespace
100 & 0.2700\\
120 & 0.7500\\
140 & 1.8500\\
160 & 4.2000\\
180 & 8.8000\\
\bottomrule
\end{tabular}
\end{table}

\hypertarget{parts}{%
\chapter{Parts}\label{parts}}

You can add parts to organize one or more book chapters together. Parts can be inserted at the top of an .Rmd file, before the first-level chapter heading in that same file.

Add a numbered part: \texttt{\#\ (PART)\ Act\ one\ \{-\}} (followed by \texttt{\#\ A\ chapter})

Add an unnumbered part: \texttt{\#\ (PART\textbackslash{}*)\ Act\ one\ \{-\}} (followed by \texttt{\#\ A\ chapter})

Add an appendix as a special kind of un-numbered part: \texttt{\#\ (APPENDIX)\ Other\ stuff\ \{-\}} (followed by \texttt{\#\ A\ chapter}). Chapters in an appendix are prepended with letters instead of numbers.

\hypertarget{footnotes-and-citations}{%
\chapter{Footnotes and citations}\label{footnotes-and-citations}}

\hypertarget{footnotes}{%
\section{Footnotes}\label{footnotes}}

Footnotes are put inside the square brackets after a caret \texttt{\^{}{[}{]}}. Like this one \footnote{This is a footnote.}.

\hypertarget{citations}{%
\section{Citations}\label{citations}}

Reference items in your bibliography file(s) using \texttt{@key}.

For example, we are using the \textbf{bookdown} package (Xie 2021) (check out the last code chunk in index.Rmd to see how this citation key was added) in this sample book, which was built on top of R Markdown and \textbf{knitr} (Xie 2015) (this citation was added manually in an external file book.bib).
Note that the \texttt{.bib} files need to be listed in the index.Rmd with the YAML \texttt{bibliography} key.

The RStudio Visual Markdown Editor can also make it easier to insert citations: \url{https://rstudio.github.io/visual-markdown-editing/\#/citations}

\hypertarget{blocks}{%
\chapter{Blocks}\label{blocks}}

\hypertarget{equations}{%
\section{Equations}\label{equations}}

Here is an equation.

\begin{equation} 
  f\left(k\right) = \binom{n}{k} p^k\left(1-p\right)^{n-k}
  \label{eq:binom}
\end{equation}

You may refer to using \texttt{\textbackslash{}@ref(eq:binom)}, like see Equation \eqref{eq:binom}.

\hypertarget{callout-blocks}{%
\section{Callout blocks}\label{callout-blocks}}

The R Markdown Cookbook provides more help on how to use custom blocks to design your own callouts: \url{https://bookdown.org/yihui/rmarkdown-cookbook/custom-blocks.html}

\hypertarget{sharing-your-book}{%
\chapter{Sharing your book}\label{sharing-your-book}}

\hypertarget{publishing}{%
\section{Publishing}\label{publishing}}

HTML books can be published online, see: \url{https://bookdown.org/yihui/bookdown/publishing.html}

\hypertarget{pages}{%
\section{404 pages}\label{pages}}

By default, users will be directed to a 404 page if they try to access a webpage that cannot be found. If you'd like to customize your 404 page instead of using the default, you may add either a \texttt{\_404.Rmd} or \texttt{\_404.md} file to your project root and use code and/or Markdown syntax.

\hypertarget{metadata-for-sharing}{%
\section{Metadata for sharing}\label{metadata-for-sharing}}

Bookdown HTML books will provide HTML metadata for social sharing on platforms like Twitter, Facebook, and LinkedIn, using information you provide in the \texttt{index.Rmd} YAML. To setup, set the \texttt{url} for your book and the path to your \texttt{cover-image} file. Your book's \texttt{title} and \texttt{description} are also used.

This \texttt{gitbook} uses the same social sharing data across all chapters in your book- all links shared will look the same.

Specify your book's source repository on GitHub using the \texttt{edit} key under the configuration options in the \texttt{\_output.yml} file, which allows users to suggest an edit by linking to a chapter's source file.

Read more about the features of this output format here:

\url{https://pkgs.rstudio.com/bookdown/reference/gitbook.html}

Or use:

\begin{Shaded}
\begin{Highlighting}[]
\NormalTok{?bookdown}\SpecialCharTok{::}\NormalTok{gitbook}
\end{Highlighting}
\end{Shaded}

\cleardoublepage
    \bookmarksetupnext{level=part}
    \phantomsection
    \addcontentsline{toc}{chapter}{References}
    \begin{centering}
    REFERENCES\\
    \vskip 1 \baselineskip
    \end{centering}

\hypertarget{refs}{}
\begin{CSLReferences}{1}{0}
\leavevmode\vadjust pre{\hypertarget{ref-xie2015}{}}%
Xie, Yihui. 2015. \emph{Dynamic Documents with {R} and Knitr}. 2nd ed. Boca Raton, Florida: Chapman; Hall/CRC. \url{http://yihui.org/knitr/}.

\leavevmode\vadjust pre{\hypertarget{ref-R-bookdown}{}}%
---------. 2021. \emph{Bookdown: Authoring Books and Technical Documents with r Markdown}. \url{https://CRAN.R-project.org/package=bookdown}.

\end{CSLReferences}


% Index?

\end{document}
